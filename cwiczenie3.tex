\documentclass[a4paper]{article}
\usepackage[T1]{fontenc}
\begin {document}
\title{Ćwiczenia 3}
\author{Damian Graczyk}
\maketitle
\newpage
%%zadanie 1
Tabela 1: Przykładowy system decyzyjny (U,A,d), modelujący problem diagnozy medycznej, której efektem jest decyzja o wykonaniu lub nie wykonaniu operacji wycięcia wyrostka robaczkowego, U=\{u1,u2,...,u10\}, A=\{a1,a2\}, d->D=\{TAK,NIE\}
\begin {center}
\begin {tabular} {c | c c c}
\hline
\hline
 Pacjent & Ból brzucha & Temperatura ciała & Operacja \\
 \hline
 u1 & Mocny & Wysoka & Tak \\
 u2 & Średni & Wysoka & Tak \\
 u3 & Mocny & Średnia & Tak \\
 u4 & Mocny & Niska & Tak \\
 u5 & Średni & Średnia & Tak \\
 u6 & Średni & Średnia & Nie \\
 u7 & Mały & Wysoka & Nie \\
 u8 & Mały & Niska & Nie \\
 u9 & Mocny & Niska & Nie \\
 u10 & Mały & Średnia & Nie \\
 \hline
 \hline
\end {tabular}
\end {center}
\newpage
%%zadanie 2
\begin{center}
\begin{tabular} { c c | c c c c c c}
 A & B & NOT & AND & NAND & OR & NOR & XOR \\
 \hline
 0 & 0 & 0 & 0 & 1 & 0 & 1 & 0 \\
 0 & 1 & 1 & 0 & 1 & 0 & 0 & 1 \\
 1 & 0 & 1 & 0 & 1 & 0 & 0 & 1 \\
 1 & 1 & 0 & 1 & 0 & 1 & 0 & 0
\end{tabular}
\end{center}


\end {document}